%!TeX root = ../Thesis.tex
\documentclass[../Thesis.tex]{subfiles}



\begin{document}

\begingroup
\clearpage% Manually insert \clearpage
\let\clearpage\relax% Remove \clearpage functionality
\vspace*{-2cm}% Insert needed vertical retraction



\chapter*{Abstract}
\endgroup
\addcontentsline{toc}{chapter}{Abstract} 

    Surviving remnants from binary neutron star mergers, remnants that have not yet collapsed to a black hole,  are some of the most extreme forms of matter known in the Universe.    
    The densities of surviving merger remnants are $\sim 10^{14}-10^{15}\,\mathrm{g\, cm^{-3}}$ which is up to several times nuclear density.
    The temperature of the post-merger remnant can reach values from tens to hundreds of $\mathrm{MeV}$.
    This combination of extreme temperatures and densities result in a state of matter that is observable only in proto-neutron stars and post-merger remnants.
    Although binary neutron star post-merger remnants have yet to be directly observed, they are a target for gravitational-wave detection in the near future. \par

    A surviving post-merger remnant could potentially be detected by applying a matched filter between an incoming gravitational-wave signal and the gravitational-wave strain generated from numerical-relativity simulations.
    However, these simulations are performed in full general relativity and are extremely expensive.
    This prohibits their direct use in the detection and parameter estimation of post-merger remnant properties.
    Other methods are required to generate post-merger gravitational waveforms that are computationally affordable. \par
    
    Here we develop two methods that can generate gravitational waveforms of post-merger remnants.
    In the first method, we train a hierarchical model to find a relationship between the progenitor neutron star properties and gravitational-wave spectra produced by numerical-relativity simulations.
    After training, we generate gravitational-wave spectra in a fraction of a second and measure cross-validated fitting-factors with a mean of 0.95 where 1.0 is a perfect fit.\par
    
    For the second method, we introduce an analytical model that can successfully capture the richness of post-merger gravitational waveforms generated by numerical-relativity simulations.
    We measure median fitting factors between the gravitational-wave strain produced by numerical-relativity simulations and our inferred waveforms as $> 0.9$.
    With this method, we can detect surviving post-merger remnants with signal-to-noise ratios of $\geqslant 7$.
    We find that, at signal-to-noise ratios of 15, the dominant post-merger oscillation frequency can be constrained to $\pm_{1.2}^{1.4}\%$, and the tidal coupling constant can be constrained to $\pm^{9}_{12}\%$. \par

    We  then measure when the post-merger remnant will collapse into a black hole.
    We show that we need a gravitational-wave detector network of Einstein Telescope with Cosmic explorer to measure a collapse time of $\sim\!10$\,ms for a GW170817-like event at $\sim\!40$\,Mpc.
    These three methods introduce additional tools that allow for the detection and parameter estimation of post-merger remnants when gravitational-wave instruments achieve sufficient sensitivity.
    
    
\begingroup
\clearpage% Manually insert \clearpage
\let\clearpage\relax% Remove \clearpage functionality
\vspace*{-2cm}% Insert needed vertical retraction



\chapter{Introduction} \label{chapter:Introduction}
\endgroup 

    \section{Gravitational waves} 
    In 2017, the LIGO and Virgo collaborations observed the first direct gravitational waves from a binary neutron star merger, GW170817~\cite{GW170817Detection}.
    This began the era of multi-messenger gravitational-wave observations with coincident detections in many parts of the electromagnetic spectrum~\cite{Goldstein2017,Coulter2017,Troja2017,Nicholl2017,Chornock2017,Margutti2017,Alexander2017,GW170817multi}. \par
    
    Gravitational waves are a phenomenon predicted by general relativity which was formulated more than 100 years ago by Albert Einstein with the famous formula
    \begin{align}
        G_{\mu\nu} & = R_{\mu\nu} - \frac{1}{2}R  g_{\mu\nu}  =  \frac{8\pi G}{c^4} T_{\mu\nu}, \label{eq:Intro:GeqT}
    \end{align}
    where $G_{\mu\nu}$ is the space-time curvature tensor and $T_{\mu\nu}$ is the stress energy tensor. The local metric tensor is $g_{\mu\nu}$, $R_{\mu\nu}$ is the Ricci tensor and $R$ is the Ricci scalar. 
    The gravitational constant is $G$ and the speed of light in vacuum is $c$, though for remainder of this work we will use geometric units of $G=c=1$ unless otherwise stated. 
    Equation~\ref{eq:Intro:GeqT} demonstrates how the curvature of the space-time, $G_{\mu\nu}$, alters the dynamics of matter, $T_{\mu\nu}$, and vice-versa. \par
    Movement of matter in space can generate gravitational waves.
    For simulation purposes, the movement of matter can be modelled by the Einstein field equations~Eq.(\ref{eq:Intro:GeqT}) and perturbations of black holes~\cite[e.g.,][]{Moncrief1974,Abrahams1996,Andrade1999,Nagar2005}.
    This methodology allows the calculation of multipole functions of odd ($\mathcal{Q}^\times_{lm}$) and even ($\mathcal{Q}^+_{lm}$) parity, with angular indices of $(l,m)$.
    These multipoles can be evaluated by numerical-relativity simulations which allow the extraction of gravitational waves.
    The plus, $h_{+}$, and cross, $h_{\times}$, polarisations of the gravitational waves are related to the multipoles as follows~\cite{Moncrief1974,Abrahams1996,Andrade1999,Nagar2005,Baiotti2008}:\footnote{Please note that $\mathcal{Q}^+_{lm}$ and $\mathcal{Q}^\times_{lm}$ have different units, see Ref.~\cite{Nagar2005} for more information.}
     \begin{align}
        h_{+} - i h_{\times} & = \frac{1}{r\sqrt{2}}\sum_{l,m} {}_{-2}Y^{lm}\left(\mathcal{Q}^{+}_{lm} -i \int_{-\infty}^t\mathcal{Q}^\times_{lm}(t')dt'\right). \label{eq:Intro:MultipoleExpansion}
    \end{align}   
    Here, $r$ is the distance to the gravitational-wave source and ${}_{-2}Y^{lm}$ are the spin-weighted spherical harmonics with spin $s=-2$ and angular indices of $(l,m)$.
    For most sources, gravitational waves are predominantly emitted in the $l=m=2$ mode. 
    In this case Eq.(\ref{eq:Intro:MultipoleExpansion}) evaluates to
     \begin{align}
        h_{+} + i h_{\times} & = \frac{1}{2 r} \sqrt{\frac{5}{2\pi }} e^{2 i \phi } \cos ^4\left(\frac{\theta }{2}\right)\left(\mathcal{Q}^{+}_{22} -i \int_{-\infty}^t\mathcal{Q}^\times_{22}(t')dt'\right), \label{eq:Intro:MultipoleExpansionlm22}
    \end{align}
    where  $\theta$ is the polar angle and $\phi$ is the azimuthal angle usually defined relative to orbital angular momentum vector. \par
    
    The energy loss due to gravitational waves can be calculated from Eq.(\ref{eq:Intro:MultipoleExpansion}) as~\cite{Nagar2005}
     \begin{align}
        \frac{dE}{dt} & = \frac{1}{32 \pi}\sum_{l,m} \left(\left|\frac{\mathrm{d} \mathcal{Q}^{+}_{lm}}{dt}\right|^2 + \left|\mathcal{Q}^\times_{lm}\right|^2\right). \label{eq:Intro:EnergyLoss}
    \end{align}
    For binary systems, this results in an incremental reduction in the orbital separation which was famously measured in the Hulse-Taylor binary neutron star, B1913+16~\cite{Hulse1975,Taylor1982,Weisberg2016}. \par 
    
    If the separation between binary compact objects is small enough, then this loss of energy can lead to a merger.
    This was the case for the first direct detection of a binary black hole merger, GW150914, and first neutron star merger, GW170817~\cite{GW150914Detection,GW170817Detection}. 
    The decrease in the orbital separation and increase in gravitational-wave amplitude, results in a characteristic `chirp' signal for the inspiral of binary compact objects.
    
    Assuming a circular orbit of two point masses of mass $M_1$ and $M_2$, the Newtonian quadrupole formula can be used to calculate the evolution of the orbital separation, $a$, due to gravitational wave emission:~\cite[e.g.,][]{Cutler1994}
    \begin{align}
        \frac{da}{dt} & = -\frac{64}{5}\frac{\mu M^2}{a^3}, \label{eq:Intro:OrbitalEvolution}        
    \end{align}
    where  $M \equiv M_1 + M_2$ is the total mass, and $\mu \equiv M_1 M_2 / M$ is the reduced mass.
    The time evolution of gravitational-wave strain frequency can then be calculated as
    \begin{align}
        \frac{df}{dt} & = \frac{96}{5}\pi^{8/3}\mathcal{M}^{5/3}f^{11/3}, \label{eq:Intro:FreqEvolution}        
    \end{align}   
    where $\mathcal{M} \equiv \mu^{3/5} M^{2/5}$ is the chirp mass of the binary system. 
    Using Eqs.(\ref{eq:Intro:OrbitalEvolution}\Hyphdash*\ref{eq:Intro:FreqEvolution}), an expression for the frequency and the gravitational-wave strain magnitude can be calculated~~\cite[e.g.,][]{Cutler1994}: 
      \begin{align}
      f &  \propto \mathcal{M}^{- 5/8}(t_c-t)^{- 3/8},\label{eq:Intro:chirp_f}\\
        |h(t)| & \propto  \frac{\mathcal{M}^{5/4}}{D}(t_c-t)^{-1/4}, \label{eq:Intro:chirp_ht} 
    \end{align}
    where $t$ is the time, $t_c$ is the time until coalescence, and $D$ is the luminosity distance to the source. 
    The combination of Eq.(\ref{eq:Intro:chirp_f})~and~Eq.(\ref{eq:Intro:chirp_ht}) result in the chirp waveform detected in the inspiral of compact object mergers.
    \par
    The first direct gravitational-wave detection from the inspiral of a binary neutron star merger, GW170817, was observed by a detector network consisting of the LIGO detectors at Hanford, Washington and Livingston, Louisiana in the USA, and the Virgo detector in Cascina, Italy~\cite{AdvancedLIGO2015,AdvancedVirgo2015,GW170817Detection}. 
    All three detectors consist of `L' shaped laser interferometers with free-falling test-masses.
    The gravitational-wave strain is found by measuring the optical path difference between the perpendicular beam tubes. \par
    
    Another binary neutron star (possibly neutron star - black hole) merger, GW190425, was detected in 2019~\cite{GW190425Detection,GWTC2,PopGWTC2}.  However, the luminosity distance of GW190425, $159^{+69}_{-72}\,$Mpc, and the broad sky localisation of the source, $8284\,\mathrm{deg}^2$, precluded any coincident electromagnetic observations~\cite{GW190425Detection}. In comparison, the sky localisation of GW170817 was $28\,\mathrm{deg}^2$ at a distance of $\sim 40\,\Mpc$.
    Therefore, to date, GW170817 is the only gravitational-wave signal that was observed with coincident electromagnetic radiation and will be discussed further in Section~\ref{sec:Intro:NeutronStarMergers}.
    
    Although the LIGO and Virgo collaborations were able to detect part of the inspiral for GW170817, the sensitivity of the existing network was not sufficient to detect the gravitational-wave strain from either the late inspiral or the post-merger remnant.
    This is because the gravitational-wave strain emitted from these phases are at high frequencies, where the existing networks lack sensitivity.
    Figure~\ref{fig:Intro:ASDs} shows the amplitude spectral sensitivity at design sensitivity for LIGO Hanford in blue and Virgo in dashed orange.
    Funding has been secured to upgrade the LIGO network to A+ sensitivity, which will result in an improvement of $~\sim 2-3$~\cite{Miller2015, NEMO2020} and is shown with the green dotted curve.
    \par
\begin{figure}[H]
    \centering
    \includegraphics[width=0.95\textwidth]{{Thesis_ASDs}.pdf}
    \caption{Amplitude spectral sensitivity curves in $\Hz^{-1/2}$ for Virgo (orange, dashed) and Hanford (blue) at advanced design sensitivity, A+ (green, dotted), NEMO (red, dashed), Einstein Telescope (purple, dash-dotted) and Cosmic Explorer (black). The NEMO detector is $\sim 5$ times more sensitive than A+ at $\sim 2\,\kHz$ frequency.} 
        \label{fig:Intro:ASDs}
\end{figure}    
    A high-frequency detector has been proposed that will focus specifically on gravitational waves from the late inspiral and the post-merger remnant~\cite{NEMO2020}.
    This detector will be known as the Neutron star Extreme Matter Observatory (NEMO) and will serve two major goals.
    The first, as mentioned above, is gravitational-wave measurements of post-merger remnants and late time mergers.
    The second goal is to develop technology that will allow the third-generation observatories, Cosmic Explorer and Einstein Telescope, to be built \cite{PSD:CE,Punturo2010}.
    Figure~\ref{fig:Intro:ASDs} also shows the amplitude spectral density for NEMO (red, dashed), Einstein Telescope (Purple, dash-dot), and Cosmic Explorer (black).
    NEMO is $\sim 5$ times more sensitive than A+ at a frequency of $\sim 2\,\kHz$, resulting in a $\sim 10-15$ times improvement when compared to Advanced LIGO. 

    \section{Neutron stars}
    \label{sec:Intro:NeutronStars}
    Neutron stars are compact stellar remnants that are formed following supernovae of stars with masses of $\gtrsim 8\,\mathrm{M}_\odot$~\cite[e.g.,][]{Lattimer2004}.
    They can also be formed from the dynamics of binary star systems~\cite[e.g.,][]{Ruiter2019,LiuX2020,LiuD2020}.
    Neutron stars are created when the gravitational collapse of the inner stellar core overcomes the electron degeneracy pressure.
    The proto-neutron star matter is then supported against further collapse by neutron (or quark) degeneracy pressure. \par
    
    Proto-neutron stars cool below the Fermi energy at nuclear density, $\mathcal{E}_\mathrm{F}\sim 50\,$MeV, in the timescale of seconds ~\cite[e.g.,][]{Pons1999,Yakovlev2001,Paschalidis2012,Perego2019}.
    Such relatively low temperatures allow regions of superconductivity and superfluidity to form within the neutron star interior.
    Neutron stars are extremely dense objects with a central density $\sim 5\Hyphdash*10$ times nuclear density with magnetic fields of $\sim 10^8\Hyphdash*10^{15}\,$G~\cite[e.g.,][]{Reisenegger2001,Lattimer2004,Manchester2004}. % Core density Lattimer2004
    They consist of the most dense bulk matter  in the Universe.
     \par
    General-relativistic corrections to Newtonian physics are necessary to determine the relationship between the neutron star mass and radius.
    For non-rotating stars, this relationship calculated by~\cite{Tolman1939,Oppenheimer1939}
   \begin{align}
        \frac{dm}{dr} & = 4\pi \rho r^2,\label{eq:Intro:TOV:mass_gradient}\\
        \frac{dp}{dr} & = -\frac{p + \rho}{r(r- 2  m)}\left(4\pi p r^3 + m \right), \label{eq:Intro:TOV:pressure_gradient} 
    \end{align}
    where $r$ is the radial distance, $m$ is the mass enclosed up to $r$, $p$ is the pressure at $r$, and $\rho$ is the density at $r$. 
    Equations~\ref{eq:Intro:TOV:mass_gradient}\Hyphdash*\ref{eq:Intro:TOV:pressure_gradient}, in combination with the equation of state, allow the the mass and radius of neutron stars to be calculated, as well as the maximum non-rotating neutron star mass, $\mtov$, for the given equation of state.
    \begin{figure}[H]
        \centering
        \includegraphics[scale=0.4]{{NSMassRadiusDiagramHu2020Cropped}.png}
        \caption
        {Mass-radius diagram for several different equations of state (Figure from Ref.~\cite{Hu2020}). The horizontal axis shows the radius in km and the vertical axis shows the neutron star mass in $\msun$. The two most massive known neutron stars, J0348+0432~\cite{Antoniadis2013} and J0740+6620~\cite{Cromartie2020}, are shown as horizontal shaded regions.} 
        \label{fig:Intro:mass-radius}
        % Unless provided in the caption above, the following copyright applies to the content of this slide: © The Author(s) 2020. Published by Oxford University Press on behalf of The Royal Astronomical Society.This is an Open Access article distributed under the terms of the Creative Commons Attribution License (http://creativecommons.org/licenses/by/4.0/), which permits unrestricted reuse, distribution, and reproduction in any medium, provided the original work is properly cited.

\end{figure}
    Figure~\ref{fig:Intro:mass-radius} shows the mass-radius relationship for different prospective neutron star equations of state~\cite{Hu2020}. 
    The $\mtov$ for a given equation of state is indicated by the mass value corresponding to the upper-most end of the mass-radius curve.
    Any additional mass accumulated beyond $\mtov$ introduces a radial instability that leads to the collapse of the neutron star into a black hole.
    The heaviest neutron star  discovered to date is J0740+6620 with mass of $2.14^{+0.10}_{-0.09}\,\msun$ (Fig.~\ref{fig:Intro:mass-radius}, pale yellow region)~\cite{Cromartie2020}.
    Any equation of state that does not enter this region can be eliminated as non-physical. \par
    
    An equation of state can further classified by its dimensionless compactness, $C\equiv GM/Rc^2$, where $M$ and $R$ are the neutron star mass and radius, respectively.
    Large values of $C$, corresponding to soft equations of state, are shown with curves towards the left of Fig.~\ref{fig:Intro:mass-radius}, whereas stiffer equations of state, with smaller $C$, are located towards the right of Fig.~\ref{fig:Intro:mass-radius}.
    The black region in Fig.~\ref{fig:Intro:mass-radius}, where $C\geqslant 0.5$, is excluded to neutron stars as the Schwarzschild radius is larger than the prospective neutron star radius.\footnote{The Schwarzschild radius, $Rs = 2GM/c^2$, for a non-rotating black hole corresponds to $C=0.5$ which corresponds to a gradient on the Mass-radius diagram of $c^2/2G \approx 0.339 \msun/\mathrm{km}$.}
    % $M[\msun]/R[km] = c^2/2G\msun$
    The grey region is excluded by causality, where the speed of sound in the neutron star exceeds $c$.
    Given these constraints, the radius of neutron stars are $\gtrsim 10\,$km with $\mtov \gtrsim 2.2\, \msun$. 
    

    \section{Neutron star mergers} \label{sec:Intro:NeutronStarMergers}
    Binary neutron star mergers emit vast amounts of energy in gravitational waves.
    The estimated radiated energy for the GW170817 merger was $\gtrsim 5\times 10^{52}\,$ergs~\cite{GW170817Detection}, equivalent to $\sim 1\%$ of the total rest mass of the two neutron stars.
    For a source at a distance of $D$ from the observer, the amplitude of gravitational waves are  $\propto 1/D$, whereas the amplitude of electromagnetic waves are $\propto 1/D^2$.
    Additionally, electromagnetic waves can be attenuated by intervening media between the source and the detector whereas gravitational waves are transparent to the intervening media.
    Hence gravitational waves are detectable over much larger distances in comparison to electromagnetic radiation with the same source luminosity. \par
    Gravitational-wave strain emitted from the inspiral of binary neutron star mergers, approximated by Eqs.(\ref{eq:Intro:FreqEvolution}\Hyphdash*\ref{eq:Intro:chirp_ht}), can be recast in the frequency domain using the stationary phase approximation~\cite{Cutler1994}
    \begin{align}
        h_{22}(f) & \equiv \frac{\mathcal{A}}{D}\mathcal{M}^{5/6}f^{-7/6} \exp(i\Psi(f)), \label{eq:Intro:StationaryPhaseApproxH22} \\
        \Psi(f) & = 2\pi f t_c - \phi_c - \frac{\pi}{4} +\frac{3}{4}\left(8\pi\mathcal{M}f\right)^{-5/3}, \label{eq:Intro:StationaryPhaseApproxPsi} 
    \end{align}
    where $h_{22}(f)$ is the Fourier transform of the $(l,m)=(2,2)$ gravitational-wave strain, and $\Psi(f)$ is the corresponding phase.
    Here, $f$ is the frequency~(Eq.(\ref{eq:Intro:FreqEvolution})), $\phi_c$ and $t_c$ are the phase and time of coalescence, and $\mathcal{A}$ is a factor that depends on the orientation of the system.\footnote{The orientation factor, $\mathcal{A}$, depends on angles intrinsic and extrinsic to the binary system. Intrinsic angles include the angles defining the relative orientation of the neutron star spins to the orbital plane. Extrinsic angles include the angles defining the relative orientation between the orbital plane and each of the detectors.}
    \redtext{
    % Equations~\ref{eq:Intro:StationaryPhaseApproxH22}\Hyphdash*\ref{eq:Intro:StationaryPhaseApproxPsi} are valid for $a \gtrsim 6M$ where $a$ is the orbital separation of the binary system and $a\sim 6M$ is an approximation of the innermost stable circular orbit~\cite{Cutler1994}.
    Equations~(\ref{eq:Intro:StationaryPhaseApproxH22})\Hyphdash*(\ref{eq:Intro:StationaryPhaseApproxPsi}) are valid until close to the merger time.
    The orbital separation, $a$, at the merger time can be approximated by the innermost stable circular orbit, where $a \sim 6M$.
    Therefore Eqns.(\ref{eq:Intro:StationaryPhaseApproxH22})\Hyphdash*(\ref{eq:Intro:StationaryPhaseApproxPsi}) are valid for $a \gtrsim 6M$~\cite{Cutler1994}.}
    Solving Eqs.(\ref{eq:Intro:OrbitalEvolution}\Hyphdash*\ref{eq:Intro:FreqEvolution}) for $a \sim 6M$ gives $f_{max}\approx c^3 ( 6^{3/2}\pi G M )^{-1}\,\mathrm{Hz}$.
    \par

    The validity for these equations can be extended to times closer to the merger by adding relativistic corrections calculated by post-Newtonian expansion and effective-one-body dynamics~\cite{Cutler1994,Blanchet1995, Buonanno1999,Droz1999,Damour2012,Yagi2013,Blanchet2014}. % to Eq.~\ref{eq:Intro:StationaryPhaseApproxPsi}
    These corrections introduce dependence on the masses, spins, and tidal deformabilities of the individual neutron stars.
    This allows the measurement of these system properties in addition to the chirp mass, given a gravitational-wave inspiral detection.
    However, the accuracy of the individual neutron star properties is significantly less than the accuracy of the chirp mass because the relativistic corrections required to measure these properties are fairly weak. 
    This can be seen by looking at the parameters measured from GW170817.

    \par
    The best measured quantities of GW170817 are the chirp mass, $\mathcal{M}=1.186\pm 0.001\,\msun$, and the total mass, $M=2.73^{+0.04}_{-0.01}\,\msun$~\cite{GW170817Properties}.
    The measurements that rely more on relativistic corrections are less constrained: the component masses are $M_1 = 1.48\pm~0.12\,\msun$ and $M_2=1.26\pm~0.10\,\msun$, at 90\% credible intervals~\cite{GW170817Properties}.\footnote{These posteriors assume low-spin priors that are expected from Galactic neutron star binaries. Posteriors were also taken assuming more agnostic spin priors, see Ref.~\cite{GW170817Properties} for further details.}
    The mass-weighted quadrupolar tidal deformability was measured as $\tilde{\Lambda}=300^{+420}_{-230}$ at 90\% credible interval~\cite{GW170817Properties}.\par
  
    A coincident short gamma-ray burst was detected 1.7\,s after the inferred merger time for the GW170817 gravitational-wave event~\cite{Goldstein2017,GW170817Detection}. 
    A subsequent search found an optical candidate, SSS17a/AT2017gfo, near galaxy NGC~4993 at a distance of $\simeq 40\,$Mpc, within 11 hours of the merger~\cite{Coulter2017}.
    % i-band Swope      
    Other searches made detections from radio to X-rays~\cite[e.g.,][]{Troja2017,Nicholl2017,Chornock2017,Margutti2017,Alexander2017}.
    The first detectable X-rays were found with the Chandra X-ray Observatory nine days after the merger, which was consistent with an off-axis short gamma-ray burst~\cite{Troja2017}. \par
    

    The two favoured scenarios for the launch of GRB 170817A involve a central engine of a black hole with an accretion torus~\cite[e.g.,][]{Metzger2018,Gill2019,Murguia-Berthier2021}, or a central engine of a highly magnetised post-merger neutron star~\cite[e.g.,][]{Yu2018}.
    Early lanthanide-poor kilonova observations from AT2017gfo suggest that the short gamma-ray burst was launched following the collapse of the post-merger remnant at $\sim 0.1-1\,\s$ after the merger~\cite{Metzger2018}.
    Modelling the mass of the kilnova ejecta suggests that the post-merger remnant collapsed at $\sim 1\,\s$~\cite{Gill2019}.
    Jet structure together with afterglow observations were used to conclude that the post-merger remnant collapsed at $\sim 1-1.7\,\s$~\cite{Murguia-Berthier2021}.
    However, a stable neutron star remnant was found to be consistent with modelling of the AT2017gfo kilonova if the kilonova was powered mostly by radioactivity early on with additional energy from the neutron star remnant after $\sim 3\,$days~\cite{Yu2018}. \par
    
    Using gravitational waves to measure if and when the post-merger remnant of GW170817 collapsed to a black hole would greatly aid the modelling of GRB 170817A/AT2017gfo.
    Unfortunately, existing gravitational-wave detectors are not sensitive enough in the $\kHz$ range (see Fig.~\ref{fig:Intro:ASDs} and Sec.~\ref{sec:Intro:StrainAndSpectra}) and the collapse time is one of the most difficult parameters to predict with numerical-relativity simulations.

    \section{Numerical relativity simulations}

    
    Numerical-relativity simulations are required for determining the dynamics of the late inspiral of the binary neutron star and the post-merger remnant.
    To perform a numerical-relativity simulation, the equation of state must be defined, a set of initial conditions must be evaluated in general relativity, and then Einstein's field equations must be evolved.
    Each of these three requirements make these simulations difficult.
    \redtext{The equation of state for cold neutron stars is still unknown, and the equation of state for hot post-merger remnants is even more uncertain,} after all, this is the hottest and densest bulk matter in the Universe.\par

    The initial conditions for binary neutron star mergers in general relativity are complex and a number of methods have been developed for this task.
    Codes that have been used for this include \texttt{LORENE}~\cite{Gourgoulhon2001} and \texttt{COCAL}~\cite{Tsokaros2015}.
    % The \texttt{LORENE} formulation assumes a conformally-flat approximation for  neutron star binaries and uses spectral decomposition to solve for the initial conditions~\cite{Gourgoulhon2001}.
    % This formulation constructs quasiequilibrium configurations which are approximations of the inspiral dynamics using exactly circular orbits.
    \redtext{The \texttt{LORENE} code used multi-domain spectral-methods to solve for the initial conditions of neutron star binaries assuming a conformally-flat approximation~\cite{Gourgoulhon2001}.
    This was achieved by constructing quasiequilibrium configurations, which are approximations of the inspiral dynamics using exactly circular orbits.}
    % Quasiequilibrium configurations are also generated by \texttt{COCAL}.
    % This is performed by excising regions around the neutron stars and solving Green's integral formula using multipole expansion and finite-difference methods~\cite{Uryu2012,Tsokaros2015}.
    \redtext{The \texttt{COCAL} code has also been used to generate quasiequilibrium initial conditions for binary neutron stars mergers.
    Here, the field equations are expressed in elliptic form and are solved by multipole expansion of Green's integral formula~\cite{Uryu2012,Tsokaros2015}.}
    Once the initial conditions have been approximated then the space-time can be evolved.

    Space-time evolution methods include the following formulations: Arnowitt-Deser-Misner (ADM)~\cite{Arnowitt2008}\footnote{The original book reference for ADM formalism is out of print and the authors have listed the cited paper as an `intentially unretouched' alternative.},  Baumgarte-Shapiro-Shibata-Nakamura-Oohara-Kojima (BSSNOK)~\cite{Nakamura1987,Shibata1995,Baumgarte1999,Alcubierre2000} and Z4c~\cite{Bona2003,Bona2004,Gundlach2005,Bernuzzi2010,Hilditch2013}. 
    The ADM formalism is a Hamiltonian method that splits the Einstein's field equations into separate time and space components.
    This is achieved by applying a gauge that consists of a shift vector and lapse function~\cite{Arnowitt2008}.
    % Unfortunately, ADM numerical simulations were found to be unstable to accumulated errors over the simulation time~\cite{Brandt2000,Kidder2001,Alcubierre2001a} and a number of different formulations were developed to improve this.
    \redtext{Unfortunately, as ADM numerical simulations were weakly hyperbolic in some guages, errors were found to accumulate over the simulation time~\cite{Brandt2000,Kidder2001,Alcubierre2001a}.
    To counter this, a number of different formulations were devised to improve this.}  \par
    
    To aid computation convergence, the BSSNOK method extended the ADM formalism by introducing a conformal factor to the three-metric and selecting a dissipative hyperbolic driver for the  shift vector~\cite{Nakamura1987,Shibata1995,Nakamura1999,Shibata1999,Baumgarte1999,Alcubierre2000,Alcubierre2001}. % Dissipative hyperbolic driver from Alcubierre2001
    The Z4c formalism modified the BSSNOK method by adding a four-vector of constraints, $Z_\mu$, to the Einstein field equations~\cite{Bona2003,Bona2004,Gundlach2005,Bernuzzi2010,Hilditch2013}. 
    This resulted in a propagating Hamiltonian constraint which helps to reduce constraint-violation growth by damping all constraint violations  except for constant modes~\cite{Gundlach2005}.
    % Constraint violations can be more easily tracked as non-vanishing values of $Z_\mu$ though particular attention needs to be paid to reflection of constraint violations at boundaries~\cite{Bernuzzi2010,Ruiz2011,Hilditch2013}.  \par
    \redtext{The introduction of $Z_\mu$, given an approapriate gauge, enforces hyperbolic solutions of the Einstein field equations by placing additional constraints on the system, thereby improving convergence~\cite{Bernuzzi2010,Ruiz2011,Hilditch2013}.} \par
    

    
    General-relativisitic hydrodynamic methods include codes such as \texttt{Whisky}~\cite{Baiotti2005}, \texttt{WhiskyTHC}~\cite{Baiotti2005,Radice2014,Radice2014b,Radice2015}, and \texttt{BAM}~\cite{Brugmann2004,Brugmann2008,Thierfelder2011,Dietrich2015,Bernuzzi2016b,Dietrich2019a}. 
    \texttt{WhiskyMHD} is a magneto-hydrodynamic extension to \texttt{Whisky}~\cite{Giacomazzo2007,Giacomazzo2011,Giacomazzo2013}, allowing for the introduction of magnetic fields.
    In this work we use gravitational-wave signals extracted from general-relativisitic hydrodynamic simulations from Refs.~\cite{Bernuzzi2014,Rezzolla2016,Radice2016,Dietrich2017b,Radice2017,Radice2017a,Radice2018,Dietrich2018}. 
    Refs.~\cite{Bernuzzi2014,Dietrich2017b} use the \texttt{BAM}  evolution code with the Z4c formulation, Refs.~\cite{Radice2016,Radice2017,Radice2017a,Radice2018} use the \texttt{WhiskyTHC} evolution code with the Z4c formulation, and Ref.~\cite{Rezzolla2016} uses the \texttt{Whisky} evolution code with the BSSNOK formulation. \par 

    The aforementioned tools allow the simulation of the late-time merger and post-merger remnant.
    However, there are a number of caveats that need to be considered.
    Firstly, the simulations are expensive, for example, 32 binary neutron star simulations took  $\sim 3\times 10^6\,$CPU hrs to complete~\cite{Takami2015}.
    Because of this, the direct use of numerical-relativity simulations in detection and parameter estimation, where we need large quantities of waveforms, is unfeasible. \par
    Secondly, numerical-relativity simulations unavoidably include numerical viscosity.
    \redtext{Early simulations show that numerical viscosity could alter the properties of both the neutron star core and the surface~\cite{Shibata2000bar,Cerda-Duran2010,Bernuzzi2012,Bernuzzi2012b}.}
    Numerical viscosity was found to introduce phase errors in the late-stage merger and improvements in numerical methods were subsequently introduced to minimise these errors~\cite{Radice2014}.
    The effects of numerical viscosity on the post-merger remnant is less clear, though it was found that numerical viscosity could alter both the neutrino luminosity~\cite{Sekiguchi2016} and the presence of one-armed spiral instability~\cite{Radice2016}.\par

    Thirdly, because of the complexity of the full general-relativity simulations, it is difficult to include all the necessary physics into simulations.
    Most simulations are performed with general-relativistic hydrodynamics, some also incorporate neutrino effects~\cite[e.g.,][]{Sekiguchi2011,Perego2014,Foucart2016,Radice2016a,Zappa2018}, viscous effects~\cite[e.g.,][]{Shapiro2000,Duez2004, Radice2017,Shibata2017}, and magnetic fields~\cite[e.g.,][]{Duez2006,Duez2006a,Siegel2013,Siegel2014,Giacomazzo2011,Kiuchia2012,Giacomazzo2013,Kiuchi2014,Giacomazzo2015}. \par

    Finally, simulations of the remnant are dependent on the spatial resolution.
    Generally, increasing the spatial resolution leads to more accurate simulations at the cost of simulation time.
    Ref.~\cite{Dietrich2018} state that their post-merger simulations from  the \texttt{CoRe} database are accurate enough to infer the energy and frequency content of the post-merger remnant.
    This is an important point: the uncertainty in the phase evolution  and collapse time of the post-merger remnant, which is discussed further in Sec.~\ref{sec:Intro:StrainAndSpectra}, has had a significant impact on the design of the models developed in this thesis.

 
    \section{Post-merger remnants} \label{sec:Intro:PMR}


    Neutron star mergers can result in four outcomes, depending on the total mass of the post-merger remnant and the maximum non-rotating neutron star mass, $\mtov$.
    The four different merger remnants in descending mass order are: black holes, hypermassive neutron stars, supramassive neutron stars, and stable neutron stars.
    With a total mass of $\gtrsim 1.5\,\mtov$, a black hole is promptly formed~\cite{Weih2018}.
    A hypermassive neutron star is formed when the mass is between approximately $1.2\,\mtov$ and $1.5\mtov$~\cite{Breu2016,Weih2018}.
    A supramassive neutron star is formed if the mass is between $1.0\,\mtov$ and $1.2\mtov$, and a stable neutron star is formed when the total mass is $\leq 1.0\,\mtov$.
    A remnant that has not collapsed into a black hole is referred to as a surviving remnant.\par
    
    Newly-formed post-merger remnants are differentially rotating unless they promptly collapse to a black hole within $\sim 2\,$ms.
    The differential rotation is quenched by gravitational-wave emission, magnetic braking, viscous forces, and neutrino cooling.
    If the remnant is a hypermassive neutron star then it will collapse into a black hole when the differential rotation is reduced below a critical value~\cite{Weih2018} and thermal support is lost~\cite[e.g.,][]{Bauswein2010,Kaplan2014}.  \par
    
    If the differentially-rotating remnant is a supramassive, or stable neutron star, then the remnant will evolve into a rigidly-rotating neutron star.
    A supramassive neutron star is supported against collapse by rigid rotation and will not collapse until the angular momentum drops below a critical value~\cite{Breu2016}.
    If the remnant has a mass corresponding to a stable neutron star, then it will remain stable for all levels of rigid rotation.\par
    
    Given a post-merger remnant mass, the deciding factor that discriminates between supramassive neutron stars and hypermassive neutron stars is $\mtov$.
    References~\cite{Margalit2017,Rezzolla2018,Khadkikar2021} found upper limits on $\mtov$ of $\lesssim 2.2\,\msun$ using numerical-relativity simulations with multi-messenger observations of GW170817~\cite{Margalit2017,Rezzolla2018}, and finite temperature modelling of matter~\cite{Khadkikar2021}. 
    Upper limits of $\lesssim 2.3\,\msun$ for $\mtov$ were found using angular momentum and energy considerations~\cite{Shibata2019}, and X-ray afterglow modelling~\cite{Sarin2020}, respectively. \par

    Using $\mtov\approx 2.3\,\msun$ as a fiducial value, which is also consistent with the mass range of J0740+6620~\cite{Cromartie2020}, sets the threshold between supramassive and hypermassive neutron stars at $\approx 2.8\,\msun$. 
    This is very close to the total mass of the progenitor neutron stars for GW170817, $2.73^{+0.04}_{-0.01}\,\msun$~\cite{GW170817Properties},  which suggests that, on gravitational-wave data alone, the remnant of GW170817 could have been either a hypermassive neutron star or a supramassive neutron star.\par
    
    Most binary neutron star simulations have been performed with equal-mass, or near equal-mass, progenitors.
    Immediately after the merger, for a surviving remnant, the two inner cores of the progenitor neutron stars collide and bounce, while the surrounding material is differentially rotating~\cite{Takami2015}.
    Shock heating increases the temperature of the remnant up to $\sim 100\,$MeV~\cite{Perego2019}, exceeding the neutrino trapping temperature of 5-10\,MeV~\cite{Alford2018b}.   
    The remnant continues to differentially rotate as the cores merge in $\sim 2-5\,$ms.
    During the differential-rotation stage, the remnant is strongly emitting gravitational waves through f-mode oscillations (see Sec.~\ref{sec:Intro:StrainAndSpectra}).
 \par
    
    One picture for the evolution of differentially-rotating remnants can be drawn from Ref.~\cite{Kaplan2014}, where they show the influence of thermal pressure on equilibrium models for a $2.9\,\msun$ remnant with temperatures ranging up to 40\,MeV.
    The authors find a maximum core density where the remnant collapses which depends on the amount of differential rotation and is independent of temperature.
    They also show that the remnant reaches the critical density by losing gravitational mass while approximately conserving the baryonic mass.
    The gravitational mass is lost by: secular evolution with constant differential rotation, reduction in the amount of differential rotation, and loss of thermal support.
    This additional thermal support extends the lifetime of the surviving post-merger remnant~\cite[e.g.,][]{Bauswein2010}.
 \par
    
    Reference~\cite{Paschalidis2012} estimated important timescales for the evolution of the post-merger remnant.
    The cooling timescale, estimated by assuming that neutrinos are trapped inside the post-merger remnant and escape by diffusion, is given by
    \begin{align}
        t_{\mathrm{cool}} & \approx 770 \left(\frac{M}{2.7\,\msun}\right)\left(\frac{R}{10\,\km}\right)^{-1}\left(\frac{E_\nu}{10\,\MeV}\right)^2\,\mathrm{ms}, \label{eq:Intro:NeutrinoCooling}        
    \end{align}    

    where $E_\nu$ is the typical neutrino energy.
    We estimate a cooling time of $\sim 7$\,s by using a typical neutrino energy of $\sim 30\,\MeV$, found in numerical-relativity simulations that implement neutrino trapping~\cite{Sumiyoshi2020}, though this is reasonably uncertain due to the scaling with $E_\nu^2$.\par
    

    The timescale that sets the loss of angular momentum due to gravitational waves is given by~\cite{Paschalidis2012}
    \begin{align}
        t_{\mathrm{GW}} & \approx 150 \left(\frac{M}{2.7\,\msun}\right)^{-1}\left(\frac{R}{10\,\km}\right)^{-2}\left(\frac{f_2}{3.3\,\kHz} \right)^{-4}\left(\frac{e}{0.2} \right)^{-2}\,\mathrm{ms}, \label{eq:Intro:GWAngularMomentumLoss}        
    \end{align}
    where $f_2$ is the dominant post-merger oscillation frequency (see Sec.~\ref{sec:Intro:StrainAndSpectra}) and $e\equiv(a-b)/R$ is the ellipticity of the hypermassive neutron star. 
    Here, $a$ and $b$ are the semi-major and semi-minor axes of the neutron star, respectively, such that $R=(a+b)/2$.
    The reference value, $f_2=3.3\,\kHz$, was inferred from our model in Chapter~\ref{chapter:DetectionPE} for numerical-relativity simulation of SLy equation of state with equal-mass $1.35\,\msun$ neutron stars~\cite{Radice2016}.
    We make rough estimate for the ellipticity,  $e\approx0.2$, from the time evolution of the density profile for numerical-relativity simulation of LS220 equation of state with equal-mass $1.35\,\msun$ neutron stars~\cite{Radice2017}.\par
 
    Most postmerger numerical-relativity simulations last for only tens of milliseconds, though some longer simulations have been performed between 50-100\,ms in length~\cite{Rezzolla2010,Ciolfi2017,Ciolfi2019,DePietri2020}.
    Although Eq.(\ref{eq:Intro:GWAngularMomentumLoss}) shows an estimated emission time for gravitational waves of $\sim 150\,$ms, the uncertainty in the equation of state could place $f_2$ as low as $\sim 2.2\,\kHz$ and some simulations show that the ellipticity reduces over time~\cite{Takami2015,Shibata2017a}.
    Given these possibilities, a four-fold decrease in $e$ together with a reduction in $f_2$ to $2.2\,\kHz$ would result in $t_{\mathrm{GW}} \sim 12\,$s.
    \redtext{Although the numerical-relativity simulations do not cover enough time period after the merger, they are vital for extracting the gravitational-wave strain in the early post-merger stage where the amplitude of the strain is at its loudest.}

    

    




    \section{Gravitational-wave strain and spectra} \label{sec:Intro:StrainAndSpectra}

    
    Figure~\ref{fig:Intro:GWStrain} shows a typical gravitational-wave strain from a numerical-relativity simulation.
    The left panel shows the inspiral and post-merger components of the gravitational-wave strain for the plus polarisation, and the right panel shows a zoom of the post-merger component only.
    The numerical-relativity simulation uses the SLy equation of state with equal-mass $1.35\,\msun$ neutron stars ~\cite{Radice2016,Dietrich2018}.
    The gravitational-wave strain has a post-merger signal to noise ratio of 50 for a three detector network of LIGO Hanford, Livingston, and Virgo at advanced design sensitivity.
    The merger time is defined where $h_+(t)^2 +h_\times(t)^2$ reaches its first maximum.
    The inspiral gravitational wave $(t<0)$ shows a chirping sinusoid, increasing with frequency and amplitude until the neutron stars merge.
    The post-merger gravitational-wave strain $(t>0)$  has a much higher frequency than the inspiral component and has more structure in the frequency domain (see below).
    As mentioned in Sec.~\ref{sec:Intro:PMR}, the gravitational-wave strain in the first $\sim 2-5\,$ms, is more erratic due to  interactions of the two inner cores. 
    The interactions of the inner cores settle down after this time resulting in a somewhat monotonic f-mode oscillation in the gravitational-wave strain. 
    \par
\begin{figure}[H]
    \centering
    \begin{minipage}{0.49\textwidth}
        \centering
\includegraphics[width=\textwidth]{{Thesis_inspiral_SLy_NRonly_plus_time}.pdf}

    \end{minipage}\hfill
    \begin{minipage}{0.49\textwidth}
        \centering
        \includegraphics[width=\textwidth]{{Thesis_SLy_NRonly_plus_time}.pdf}
    \end{minipage}
        \caption{Plus polarisation of the gravitational-wave strain for inspiral and post-merger (left), and post-merger only (right). Generated from a numerical-relativity simulation using the SLy equation of state with equal-mass $1.35\,\msun$ neutron stars~\cite{Radice2016,Dietrich2018} for a post-merger signal-to-noise ratio of 50.} 
        \label{fig:Intro:GWStrain}
\end{figure}    
\begin{figure}[H]
    \centering
    \includegraphics[width=0.75\textwidth]{{Thesis_SLy_NRonly_plus_freq}.pdf}
    \caption{Frequency domain representation of the post-merger gravitational-wave strain shown on the right panel of Fig.~\ref{fig:Intro:GWStrain}. 
    Four main spectral peaks are visible in the gravitational-wave spectrum with the dominant peak, $\fpeak$, has the largest amplitude at $\sim 3300\,\Hz$. The sensitivity curves for Advanced LIGO and Advanced Virgo are shown as dashed black, and dotted black,  respectively.} 
        \label{fig:Intro:GWFreq}
\end{figure}     
    
    The frequency response corresponding to the post-merger gravitational-wave strain in the right panel of Fig.~\ref{fig:Intro:GWStrain} is shown in Fig.~\ref{fig:Intro:GWFreq}.
    The amplitude is  $|\tilde{h}(f)|\sqrt{f}$, where $\tilde{h}(f)$ is the Fourier transform of the gravitational-wave strain. 
    The post-merger gravitational-wave spectrum is dominated by primary post-merger frequency, $\fpeak$, the main f\Hyphdash*mode oscillation of the remnant, which is $\sim 3.3\,\kHz$ for this simulation~\cite[e.g.,][]{Shibata1992, Zhuge1994,Stergioulas2011}. \par
 
    Two additional oscillation modes were found in simulations that were caused by the interaction between the main f\Hyphdash*mode oscillation ($m=2$) and the $m=0$ radial pressure mode~\cite{Stergioulas2011}.
    This results in a modulated signal that should be present in the frequency response at frequencies $\fpeak \pm f_{m=0}$, designated as $\ftwominuszero$ and $\ftwopluszero$, respectively.
    The $\ftwominuszero$ peak was found to be particularly strong in spectra for high-mass mergers as long as prompt collapse does not occur~\cite{Bauswein2015}.
    Another oscillation mode, labelled $\fspiral$, where $\ftwominuszero < \fspiral < \fpeak$~\cite{Bauswein2015}, was found in the spectra for low-mass mergers.
    This mode was generated by two antipodal bulges at the outer edge of the merger remnant that rotated at a slower frequency than $\fpeak$.
    The antipodal bulges were, in turn, caused by strong tidal deformation in the early post-merger stage.
    For higher-mass mergers that do not experience prompt collapse, the peaks associated with $\ftwominuszero$ dominate over peaks associated with $\fspiral$ in the spectra~\cite{Bauswein2015}.
    In the mass region between these extremes, both $\fspiral$ and $\ftwominuszero$ can be present in the spectra of these post-merger remnants. \par
    
    
    In a different interpretation of the post-merger spectra, the three main peaks were designated as $(f_1, f_2, f_3)$ in ascending frequency with $f_2 = \fpeak$~\cite{Takami2014,Takami2015,Rezzolla2016}.
    A quasi-universal relationship was found between $f_1$ and a number of progenitor system properties, including the compactness $C$ and the dimensionless quadrupolar tidal deformability $\Lambda$~\cite{Takami2015}.
    The spectral peak, $f_1$, was found to coincide with $\fspiral$ in spectrograms for stiff equations of state but was different otherwise~\cite{Rezzolla2016}.
    The power content associated with $\ftwominuszero$ was found to be extremely small in the spectrograms  even though $\ftwominuszero$ can easily be measured in simulations.
    References~\cite{Bauswein2019a,Vretinaris2019} support the interpretation of Ref.~\cite{Bauswein2015}, citing that a larger simulation set shows that $f_1$ is actually a combination of $\fspiral$ or $\ftwominuszero$ whereas a smaller simulation set, as in Refs.~\cite{Takami2014,Takami2015,Rezzolla2016}, is not large enough to highlight this.
\par
 
    
    
    Two models of the time-domain gravitational-wave strain were used for the basis for our model in Chapter~\ref{chapter:DetectionPE}.
    The first model used a third-order exponentially-damped sinusoid and successfully matched a simulation using DD2 equation of state with equal-mass $1.35\,\msun$ neutron stars~\cite{Bauswein2016}.
    The second model implemented a quadratic frequency-drift term over time for $\fpeak$ and used this model  to measure the neutron star radius~\cite{Bose2018}.
    Spectrograms from Refs.~\cite{Rezzolla2016,Dietrich2017,Dietrich2017a} and to a lesser extent from Ref.~\cite{Maione2017}, show some frequency drifts over time in some of the frequency peaks, particularly in $\fpeak$ which lends support to the implementation of a frequency-drift term in a post-merger gravitational-wave strain model.
      \par
    
    With this in mind, we create a hybrid model of Refs.~\cite{Bauswein2016} and \cite{Bose2018} which we use in Chapters~\ref{chapter:DetectionPE} and~\ref{chapter:CollapseTime}. 
    We implement a linear frequency-drift term, as opposed to the quadratic frequency-drift term used in Ref.~\cite{Bose2018}.
    Similarly to Ref.~\cite{Bauswein2016}, we use an identical third-order system, to which we add the linear frequency-drift term.
    The model, outlined in full in Chapter~\ref{chapter:DetectionPE}, is $H w_j \exp(-t/T_j) \cos \left(2\pi f_j t\left[1+\alpha_j t\right]+\psi_j \right)$.
    The success of this model is indicated by fitting factors between posterior waveforms and the numerical-relativity simulation of $\gtrsim 0.92$ with most values $\sim 0.95$.
    \par
    
    A number of challenges are presented by the gravitational-wave strain generated with numerical-relativity simulations.
    Phase errors between successive resolutions accumulate over time and it is not unusual to see phase differences of $\sim 1$\,rad by the time of the merger for the finest resolutions~\cite[e.g.,][]{Radice2014b}.
    Reference~\cite{Takami2015} examined the role of the spatial resolution on the instantaneous phase of the gravitational-wave strain and found a phase difference of $0.9$\,rad between the the finest resolution tests.
    This results in a loss of $\sim 40\%$ of the signal-to-noise compared to a perfectly matched signal when using a matched-filter detection. \par
    
    Another challenge presented by the gravitational-wave strain generated with numerical-relativity simulations is the collapse time of the remnant.
    As an example, consider the following simulations from the \texttt{CoRe} gravitational-wave database using the SLy equation of state~\cite{Dietrich2018,Radice2016}.
    These simulations are designated as THC:0036:R0$x$ for $x \in [1-4]$ with grid sizes of 148\,m, 222\,m, 295\,m, and 369\,m, respectively~\cite{Radice2016}.
    The finest resolution of 148\,m has a collapse time of $\sim 15\,\ms$, the remaining three simulations have collapse times of $\sim 17, \sim 24,$ and $\sim 60\,\ms$, respectively.
    Therefore, as the resolution becomes finer to reduce the phase error in the post-merger simulations, the amount of numerical-viscosity correspondingly increases,  resulting in an earlier collapse of the post-merger remnant.
    \par
    
    It may be the case that numerical-relativity simulations will not be fully calibrated to the true dynamics of post-merger remnants until we directly measure their gravitational waves.
    This is another motivation for simple models that are able to match the gravitational-wave strain from numerical-relativity simulations while remaining flexible to variations beyond the simulation outputs.
    In particular, models that can measure the collapse time of the remnant are important as numerical-relativity simulations are unable to predict this feature well. \par

 
    Quasi-universal relationships have been found relating the properties of the post-merger remnant and the progenitor neutron stars. 
    Perturbations in general relativity were used to find relationships between the frequency of neutron star f-modes and both the neutron star density and compactness~\cite{Andersson1998b}.
    The relationship between the f-mode frequency and the neutron star density was later refined using conformally-flat solutions of the Einstein field equations, and additional relationships were found between the neutron star radius and the f-mode frequency~\cite{Bauswein2012,Bauswein2012a}.
    Simulations of perturbed slowly-rotating neutron stars highlighted relationships between the neutron star moment of inertia, the spin-induced quadrupole moment, and the dimensionless tidal deformability~\cite{Yagi2013,Yagi2013a}. \par

    Relationships were then found between the progenitor neutron star ($l=2$) dimensionless quadrupolar tidal deformability, $\Lambda$, and a number of spectral features.
    The maximum gravitational wave frequency at the time of merger, $\fmax$, was closely related to $\Lambda^{1/5}$ for equal-mass neutron stars~\cite{Read2013}.
    A linear correlation was subsequently found between $\Lambda^{1/5}$ and the dominant post-merger f-mode frequency, $\fpeak$.
    We use this linear relationship in the model discussed in Chapter~\ref{chapter:ComputingFastReliable} to re-centre the gravitational-wave spectra after alignment.

    
    \section{Post-merger gravitational-wave models} 
    \label{sec:Intro:Models}
    In this section we briefly introduce the three main papers in Chapters~\ref{chapter:ComputingFastReliable}-\ref{chapter:CollapseTime} of this thesis. 
    A common theme in all three papers is that we need methods for generating fast binary neutron star post-merger waveforms for use in the detection or parameter estimation of post-merger remnants.
    In Chapter~\ref{chapter:ComputingFastReliable} we use a model that is trained on both the gravitational-wave spectra and the neutron star system parameters to generate post-merger spectra.
    In Chapter~\ref{chapter:DetectionPE} we develop a simple analytic model to generate time-domain gravitational-wave strain and test the model with Bayesian inference on numerical-relativity simulations.
    In Chapter~\ref{chapter:CollapseTime} we develop a model to measure the collapse time of post-merger remnants and find the distance at which we can detect the post-merger collapse for different gravitational-wave detector networks.
    
    In Chapter ~\ref{chapter:ComputingFastReliable}, we develop a way to generate gravitational-wave spectra in a fraction of a second.
    We use a hierarchical model that is trained on 35 numerical-relativity spectra from equal-mass neutron star post-merger simulations.
    The progenitor neutron star system is defined by the mass of the individual neutron stars, $M$, and the equation of state is defined by the compactness of the neutron stars, $C$, and the tidal coupling constant, $\kappa_2^T$.
    These three parameters, together with the gravitational-wave spectra, are sufficient to train our model. \par
    
    The model performance is tested with leave-one-out cross-validation and we measure noise-weighted fitting-factors of $0.95\pm 0.05$ between the inferred spectra and the spectra under-test.
    This can be compared to $0.93\pm 0.05$ which is the noise-weighted fitting-factor between the spectra of two identical neutron star systems using different numerical-relativity codes.
    We also show that we can use the trained model to perform parameter estimation for $\kappa_2^T$ with the noise-weighted fitting-factor as a proxy likelihood.
    
    In Chapter \ref{chapter:DetectionPE}, we use a third-order exponentially-damped sinusoid with a linear frequency-drift to model the post-merger time-domain gravitational-wave strain.
    We perform parameter estimation with this model to generate posterior waveforms for nine post-merger numerical-relativity simulations.
    The  noise-weighted fitting-factors between the numerical-relativity simulations and the posterior waveforms are $\sim 0.95$.
    The dominant post-merger oscillation frequency, $\fpeak$, can be constrained to $\pm_{1.2}^{1.4}\%$ for a post-merger signal-to-noise ratio of 15 and $\pm_{0.2}^{0.3}\%$ for post-merger signal-to-noise ratios of 50.
    We can constrain $\kappa_2^T$ at a post-merger signal-to-noise ratio of 15 to $\pm^{9}_{12}\%$, and $\pm 5\%$ at post-merger signal-to-noise ratios of 50,  using the hierarchical model. \par

    In Chapter~\ref{chapter:CollapseTime}, we extend the model from Chapter \ref{chapter:DetectionPE} to allow the collapse of the gravitational-wave strain.
    We use post-merger gravitational-wave strain from numerical-relativity simulations of equal-mass $1.35\,\msun$ neutron stars with LS220 and SLy equations of state.
    We then perform parameter estimation on the collapse time with four different gravitational-wave interferometer configurations. \par
    
    We find that we need a network of Einstein Telescope with Cosmic Explorer to detect a post-merger that collapses $\sim10\,\ms$ after the merger.
    Two A+ detectors at design sensitivity can measure a post-merger remnant that collapses $\sim\!10\,\mathrm{ms}$  after the merger to a distance of $\lesssim 10\,\Mpc$.
    If the proposed Neutron star Extreme Matter Observatory is operational, then this distance increases to $\sim\!18\Hyphdash*26\,\Mpc$ thereby increasing the effective volume and hence detection rate by a factor of $\sim 30$.
\par



\end{document}