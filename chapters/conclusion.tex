%!TeX root = ../Thesis.tex
\documentclass[../Thesis.tex]{subfiles}

\begin{document}

\chapter{Conclusion}\label{chapter:Conclusion}
    The direct observation of gravitational waves from the binary neutron star inspiral of GW170817 and coincident detections from GRB 170817A and AT2017gfo launched the nascent field of multi-messenger gravitational-wave astronomy.
    Gravitational waves from the late inspiral and post-merger remnant of GW170817 were not detected because the gravitational-wave detectors lacked sensitivity in high frequencies.
    High frequency gravitational-wave detectors have been proposed  which may allow gravitational-wave observations of the binary neutron star post-merger remnant and late inspiral in the near-future. \par
    
    Observations of gravitational waves from the late inspiral will inform the equation of state for cold neutron stars, whereas gravitational wave detections from surviving post-merger remnants will inform the extremely dense, hot equation of state. It should be noted that, to date, equation of state investigations have been performed using the cold equation of state (e.g. see \cite{Bauswein2014,Breschi2019} for radius constraints at $\mtov$). 
    Surviving post-merger remnants are binary neutron star remnants that have not yet collapsed to a black hole.
    We need models of the post-merger gravitational-wave strain to enable observations of gravitational waves from post-merger remnants.
    Numerical-relativity simulations are the best basis for these post-merger models. \par
    
    We should not expect that future observable post-merger gravitational waves will precisely match numerical-relativity simulations, as the simulations are extremely complex.
    Compromises must be made in spatial-resolution, incorporated physics, and convergence testing, for simulations to complete in practical time-frames.
    The absolute accuracy of the phase evolution and the collapse time of the post-merger remnant are uncertain due to these compromises.\par
    
    With these computational limitations in mind, we need simple models that can be used for gravitational-wave detection and parameter estimation.
    The main purpose of this thesis was to focus on this task.
    We achieve this goal by developing a hierarchical model that can infer the progenitor neutron star system parameters and generate gravitational-wave spectra.
    We develop a simple analytical frequency-drift model which successfully matches the numerical-relativity gravitational-wave strain. 
    We modify the frequency-drift model to allow for collapsing remnants so that we can measure when the post-merger remnant collapses to a black hole.
    
    In Chapter~\ref{chapter:ComputingFastReliable}, we train our hierarchical model on 35 numerical-relativity simulations and measure leave-one-out cross-validated fitting-factors with a mean of 0.95. 
    We show that we can generate gravitational-wave spectra in a fraction of a second, given only the neutron star mass, $M$, and the tidal coupling constant, $\kappa_2^T$.
    We perform parameter estimation on $\kappa_2^T$ by using the fitting factor as a proxy for the likelihood.
    The hierarchical model is particularly suited to modelling numerical-relativity simulations as it is insensitive to phase uncertainties that can occur between simulations with adjacent spatial resolutions.

    
    In Chapter~\ref{chapter:DetectionPE},  we implement a third-order exponentially-damped sinusoid with a frequency-drift term to successfully capture the complex gravitational-wave strain from simulations of post-merger remnants.
    We find that we can achieve noise-weighted fitting-factors of $\gtrsim 0.92$, with most fitting factors of $\sim 0.95 - 0.97$.
    We use a detection threshold with a Bayes factor of $\sim 3000$ and find that a post-merger signal-to-noise ratio of $\gtrsim 7$ is needed to detect the post-merger remnant.
    We also find that the sensitivity to the uncertainty in the coalescence time is negligible for any detectable post-merger remnant.
    We calculate the dominant post-merger oscillation frequency, $\fpeak$, from the model and constrain this at 95\% credible intervals to $\sim 1.5\%$ for a post-merger signal-to-noise ratio of 15, and $\sim 0.3\%$ for a post-merger signal-to-noise ratio of 50. 
    We then use the hierarchical model trained in Chapter~\ref{chapter:ComputingFastReliable} to infer the equation of state parameters $\kappa_2^T$ and $C$.
    At 95\% confidence intervals, we constrain $\kappa_2^T$ to $\sim 12\%$ for a post-merger signal-to-noise ratio of 15, and $\sim 5\%$ for a post-merger signal-to-noise ratio of 50.
    Similarly, we constrain $C$ to $\sim 5\%$ for a signal-to-noise ratio of 15, and $\sim 2\%$ for a post-merger signal-to-noise ratio of 50.

    In Chapter~\ref{chapter:CollapseTime},   we add a collapse-time function to our frequency-drift model and measure at what distance the collapse time can be measured.
    We find that we need a combination of Einstein Telescope and Cosmic Explorer interferometers to detect a collapse time of $10\,\ms$ at a GW170817-like distance of $\sim 40\,\Mpc$.
    For a network of 2 A+ detectors, a distance of $\sim\,10\Mpc$ is required to detect a $10\,\ms$ collapse time.
    If the high-frequency NEMO detector is added to the 2 A+ network then this detection distance is increased to $\sim 25\,\Mpc$. \par
    
    These three models address the requirements for future post-merger gravitational-wave detections.
    The post-merger models consistently achieve $\gtrsim 0.90$ fitting factors, making them suitable for matched-filter detection of gravitational-wave signals.
    However, a number of enhancements can be anticipated for future work. \par
    
    An obvious model enhancement would allow for unequal-mass binaries.
    In this case, the number of available numerical-relativity simulations are significantly reduced.
    This does not exclude training a modified hierarchical model on the unequal-mass simulations, though careful attention should be  paid to the error propagation to count for the reduced number of simulations in this part of the training set.
    The frequency-drift model, with or without the collapse time extension, has yet to be tested with unequal-mass mergers.
    This could be tested and the model updated if required. \par
    
    Measurements of larger collapse times are limited by waveform systematics in the late post-merger gravitational-wave strain.
    The same waveform systematics are also present in the model used in Chapter~\ref{chapter:DetectionPE} for post-merger signals of length $\gtrsim 15\,\ms$.
    A couple of assumptions can be reassessed for future work on the frequency-drift model to help reduce these systematics. 
    The first is that the linear frequency-drift term may be too restrictive for longer signals and other analytic or semi-analytic methods could be considered.
    Secondly, the assumption of exponential damping could be reassessed, considering, for example, a power-law decay in the amplitude instead.
    \par

    The lifetimes of post-merger remnants are highly uncertain with gravitational-wave timescales ranging from $0.15-12\,\s$ for a range of ellipticities and equations of state (Eq.(\ref{eq:Intro:GWAngularMomentumLoss})).
    If the above waveform systematics are successfully addressed then detection and parameter estimation studies could be performed with much longer collapse times. 
    A successful injection study for collapse times up to $\sim 2\,\s$ may allow the measurement of collapse times for future GW170817-like events.
    This would have significantly aided model selection for the gamma-ray physics associated with GRB 170817A.\par

    For example, measuring any collapse of the post-merger remnant would rule out proposals that rely on surviving neutron star remnants~\cite[e.g.,][]{Yu2018}.
    Measuring a collapse time  of $\lesssim 0.5\,\s$ would reduce support for scenarios in Refs.~\cite{Gill2019,Murguia-Berthier2021} where collapse
    times of at least $\sim 1\,\s$ are required so that jets are launched after black hole formation. 

    A small collapse time of $\lesssim 0.05\,\s$ will rule out models that rely on the temporary survival of the post-merger remnant to account for the colour of kilonova emission~\cite{Metzger2018}.
    This highlights a few examples of how the collapse times may influence the physics of short gamma-ray bursts.\par

    The models developed in this thesis could be used to develop detection pipelines for post-merger remnants.
    Pipelines already exist that are capable of searching for post-merger remnants: \texttt{cWB}, \texttt{BayesWave}, \texttt{Viterbi}, \texttt{STAMP}, \texttt{ATrHough}, and  \texttt{FreqHough}~\cite{Thrane2011, Cornish2015,Littenberg2015,Viterbi1967,Klimenko2016, Suvorova2016,GW170817Postmerger1,Chatziioannou2017,Miller2018, Torres-Rivas2019, Oliver2019,GW170817Postmerger2}.
    A number of these tools searched for a long-lived post-merger remnant from the GW170817 merger, with searches lasting $\lesssim 500\,\s$ (\texttt{STAMP}, \texttt{cWB})~\cite{GW170817Postmerger1} and searches ranging from $\sim 2 - 24\,$hours (\texttt{Viterbi}, \texttt{STAMP}, \texttt{ATrHough}, and  \texttt{FreqHough})~\cite{GW170817Postmerger2}.
    Searches were also performed over $\lesssim 1\,\s$ (\texttt{cWB})~\cite{GW170817Postmerger1} and \texttt{BayesWave} has characterised the post-merger remnant from binary neutron star mergers~\cite{Clark2016postmerger,Chatziioannou2017,Torres-Rivas2019,Easter2020}. \par

    Pipelines developed from the analytical models in Chapters~\ref{chapter:DetectionPE}-\ref{chapter:CollapseTime} could be designed to work with signals produced from numerical-relativity simulations with lengths of $\lesssim 100\,\ms$, or long signals of $\lesssim 2\,\s$.
    Pipelines that searched for the long-lived post-merger remnant should be used for signals longer than this.
    We propose two pipelines from this work: a short pipeline with a length of $\lesssim 100\,\ms$ and a long pipeline with length  $\lesssim 2\,\s$. \par
    
    Pipelines that are of direct interest for comparison with the analytical model are \texttt{BayesWave} and  \texttt{cWB}.
    We find in Chapter~\ref{chapter:DetectionPE} that \texttt{BayesWave} is more sensitive when measuring the dominant post-merger frequency, $\fpeak$.
    \texttt{BayesWave} and the analytical model constrained $\fpeak$ to post-merger signal-to-noise ratios of $\gtrsim 9$, and $\gtrsim 15$, respectively. 
    The \texttt{cWB} pipeline is expected to be the least sensitive as it detects unmodelled coherent excess power, assuming waveform systematics have been addressed in the analytical model. \par

    The proposed pipeline would use Bayesian inference for detection and parameter estimation which will find posteriors directly related to the gravitational-wave strain of the post-merger remnant.
    This is an advantage over both \texttt{BayesWave} and \texttt{cWB}. 
    The Bayes factor would serve as a robust detection statistic in this situation due to the loud inspiral signal and the tight constraint expected on the coalescence time.
    It may be necessary to perform computational optimisation to allow the long pipeline to execute in reasonable time frames, though it may be possible to use parallel Bilby~\cite{Smith2020} to speed up this pipeline. \par
    
    The hierarchical model can only be used on the short pipeline because it is trained on numerical-relativity simulations.
    It is likely that the model will need to be extended to account for both spins and unequal-mass neutron stars, this will need to be assessed.
    The hierarchical model can be trained on as many numerical-relativity simulations as possible with careful consideration taken in selecting the spatial resolution.
    The hierarchical model would then be available to infer equation of state parameters from the posterior waveforms generated from the analytical pipeline.
\par
    Although we are not expecting binary neutron star post-merger detections in the immediate future, they may be possible if proposed high-frequency gravitational-wave detectors like NEMO are constructed, or failing that, when Einstein telescope and Cosmic Explorer detectors are operational.
    With this in mind, we must ensure that we have the tools available to fully utilise any future binary neutron star post-merger detections so that we can probe the extreme physics of these remnants.

    \par
    



    

    



\end{document}